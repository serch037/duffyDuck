\documentclass[letterpaper, 12pt, oneside]{scrartcl}

\usepackage{fancyhdr}
\usepackage{extramarks}
\usepackage{amsmath}
\usepackage{amsthm, amssymb}
\usepackage{amsfonts}
\usepackage{tikz}
\usepackage[plain]{algorithm}
\usepackage{algpseudocode}
\usepackage{enumitem}
\usepackage{fontspec}
\usepackage{microtype}
\usepackage{lmodern}
\setmainfont[Ligatures=TeX, SmallCapsFont={* Caps}, SlantedFont={* Slanted},
ItalicFeatures ={
  SmallCapsFont = {LMRomanCaps10-Oblique}
}]{Latin Modern Roman}
\setsansfont{Latin Modern Sans}
\setmonofont{Latin Modern Mono}
\setmathfont{Latin Modern Math}
\usetikzlibrary{automata,positioning}

%
% Basic Document Settings
%

\topmargin=-0.45in
\evensidemargin=0in
\oddsidemargin=0in
\textwidth=6.5in
\textheight=9.0in
\headsep=0.25in

\linespread{1.1}

\pagestyle{fancy}
% \lhead{\hmwkAuthorName}
\chead{\hmwkClass\ (\hmwkClassInstructor\ \hmwkClassTime): \hmwkTitle}
\rhead{\firstxmark}
\lfoot{\lastxmark}
\cfoot{\thepage}

\renewcommand\headrulewidth{0.4pt}
\renewcommand\footrulewidth{0.4pt}

\setlength\parindent{0pt}

%
% Create Problem Sections
%

\newcommand{\enterProblemHeader}[1]{
    \nobreak\extramarks{}{Problem \arabic{#1} continued on next page\ldots}\nobreak{}
    \nobreak\extramarks{Problem \arabic{#1} (continued)}{Problem \arabic{#1} continued on next page\ldots}\nobreak{}
}

\newcommand{\exitProblemHeader}[1]{
    \nobreak\extramarks{Problem \arabic{#1} (continued)}{Problem \arabic{#1} continued on next page\ldots}\nobreak{}
    \stepcounter{#1}
    \nobreak\extramarks{Problem \arabic{#1}}{}\nobreak{}
}

\setcounter{secnumdepth}{0}
\newcounter{partCounter}
\newcounter{homeworkProblemCounter}
\setcounter{homeworkProblemCounter}{1}
\nobreak\extramarks{Problem \arabic{homeworkProblemCounter}}{}\nobreak{}

%
% Homework Problem Environment
%
% This environment takes an optional argument. When given, it will adjust the
% problem counter. This is useful for when the problems given for your
% assignment aren't sequential. See the last 3 problems of this template for an
% example.
%
\newenvironment{homeworkProblem}[1][-1]{
    \ifnum#1>0
        \setcounter{homeworkProblemCounter}{#1}
    \fi
    \section{Problem \arabic{homeworkProblemCounter}}
    \setcounter{partCounter}{1}
    \enterProblemHeader{homeworkProblemCounter}
}{
    \exitProblemHeader{homeworkProblemCounter}
}

%
% Homework Details
%   - Title
%   - Due date
%   - Class
%   - Section/Time
%   - Instructor
%   - Author
%

\newcommand{\hmwkTitle}{Project 1}
\newcommand{\hmwkDueDate}{April 7, 2017}
\newcommand{\hmwkClass}{Calculus II}
% \newcommand{\hmwkClassTime}{Section A}
\newcommand{\hmwkClassInstructor}{Adrian Ulises Soto}
\newcommand{\hmwkAuthorName}{\textbf{Sergio Ugalde Marcano} \and \textbf{Davis
    Josh} \and \textbf{Sergio Ugalde Marcano} \and \textbf{Sergio Ugalde Marcano}  }

%
% Title Page
%

\title{
    \vspace{2in}
    \textmd{\textbf{\hmwkClass:\ \hmwkTitle}}\\
    \normalsize\vspace{0.1in}\small{Due\ on\ \hmwkDueDate\ }\\
    \vspace{0.1in}\large{\textit{\hmwkClassInstructor\ \hmwkClassTime}}
    \vspace{3in}
}

\author{\hmwkAuthorName}
\date{}

\renewcommand{\part}[1]{\textbf{\large Part \Alph{partCounter}}\stepcounter{partCounter}\\}

%
% Various Helper Commands
%

% Useful for algorithms
\newcommand{\alg}[1]{\textsc{\bfseries \footnotesize #1}}

% For derivatives
\newcommand{\deriv}[1]{\frac{\mathrm{d}}{\mathrm{d}x} (#1)}

% For partial derivatives
\newcommand{\pderiv}[2]{\frac{\partial}{\partial #1} (#2)}

% Integral dx
\newcommand{\dx}{\mathrm{d}x}

% Alias for the Solution section header
\newcommand{\solution}{\textbf{\large Solution}}

% Probability commands: Expectation, Variance, Covariance, Bias
\newcommand{\E}{\mathrm{E}}
\newcommand{\Var}{\mathrm{Var}}
\newcommand{\Cov}{\mathrm{Cov}}
\newcommand{\Bias}{\mathrm{Bias}}

\begin{document}

\maketitle

\pagebreak

\begin{homeworkProblem}
Choose a profile picture, which can be scanned from a drawing or desinged entirely on a computer. Each drawing must be
unique and must be approved before proceeding to continue the project. The drawing must have an axis of symmetry.
The design must be such that an application of Simpson's rule can be used to approximate the several integrals. Pixelated
desings are also possible. The boundary must be a simple closed curve with (no holes, no self-crossing).
\end{homeworkProblem}

\begin{homeworkProblem}
  Choose a particular scale, origin and coordinate axes for your drawing.
\end{homeworkProblem}

\begin{homeworkProblem}
  Numerically, and an approximation to the area of your drawing using Simpson's rule and the Trapezoidal rule. If your
drawing allows it, and an exact answer also.
\end{homeworkProblem}

\begin{homeworkProblem}
 Numerically, and an approximation of the perimeter of your drawing approximating it using straight lines and adding the
length.
\end{homeworkProblem}

\begin{homeworkProblem}
  Choose an axis parallel to the axis of symmetry of your gure on which you will revolve your figure to produce a life saver
with your figure as profile
\end{homeworkProblem}

\begin{homeworkProblem}
  Investigate Pappus' Theorem for solids of Revolution, and use it to
  \begin{enumerate}[label=\alph*)]
  \item Compute the volume of the resulting solid of revolution.
  \item Compute the surface area of the resulting solid of revolution
  \end{enumerate}
\end{homeworkProblem}

\begin{homeworkProblem}
 Investigate Arquimedes' Theorem to And the boyance force of the life saver assuming it is completely submerged in water,
and is filled with air.
\end{homeworkProblem}

\begin{homeworkProblem}
  Using Simpson's rule, and the volume of the solid of revolution without using Pappus' Theorem.
\end{homeworkProblem}


\begin{homeworkProblem}
  Make a real design of your life saver. You may 3d print it if you wish.
\end{homeworkProblem}

\end{document}
