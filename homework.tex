\documentclass[letterpaper, 12pt, oneside]{scrartcl}

\usepackage{fancyhdr}
\usepackage{extramarks}
\usepackage{amsmath}
\usepackage{amsthm, amssymb}
\usepackage{amsfonts}
\usepackage{tikz}
\usepackage[plain]{algorithm}
\usepackage{algpseudocode}
\usepackage{enumitem}
\usepackage{fontspec}
\usepackage{mathspec}
\usepackage{microtype}
\usepackage{lmodern}
\usepackage{graphicx}
\usepackage[english]{babel}
\setmainfont[Ligatures=TeX, SmallCapsFont={* Caps}, SlantedFont={* Slanted},
ItalicFeatures ={
  SmallCapsFont = {LMRomanCaps10-Oblique}
}]{Latin Modern Roman}
\setsansfont{Latin Modern Sans}
\setmonofont{Latin Modern Mono}
% \setmathfont{Latin Modern Math}
\usetikzlibrary{automata,positioning}

%
% Basic Document Settings
%

\topmargin=-0.45in
\evensidemargin=0in
\oddsidemargin=0in
\textwidth=6.5in
\textheight=9.0in
\headsep=0.25in

\linespread{1.1}

\pagestyle{fancy}
% \lhead{\hmwkAuthorName}
\chead{\hmwkClass\ (\hmwkClassInstructor\ \hmwkClassTime): \hmwkTitle}
\rhead{\firstxmark}
\lfoot{\lastxmark}
\cfoot{\thepage}

\renewcommand\headrulewidth{0.4pt}
\renewcommand\footrulewidth{0.4pt}

\setlength\parindent{0pt}

%
% Create Problem Sections
%

\newcommand{\enterProblemHeader}[1]{
    \nobreak\extramarks{}{Problem \arabic{#1} continued on next page\ldots}\nobreak{}
    \nobreak\extramarks{Problem \arabic{#1} (continued)}{Problem \arabic{#1} continued on next page\ldots}\nobreak{}
}

\newcommand{\exitProblemHeader}[1]{
    \nobreak\extramarks{Problem \arabic{#1} (continued)}{Problem \arabic{#1} continued on next page\ldots}\nobreak{}
    \stepcounter{#1}
    \nobreak\extramarks{Problem \arabic{#1}}{}\nobreak{}
}

\setcounter{secnumdepth}{0}
\newcounter{partCounter}
\newcounter{homeworkProblemCounter}
\setcounter{homeworkProblemCounter}{1}
\nobreak\extramarks{Problem \arabic{homeworkProblemCounter}}{}\nobreak{}

%
% Homework Problem Environment
%
% This environment takes an optional argument. When given, it will adjust the
% problem counter. This is useful for when the problems given for your
% assignment aren't sequential. See the last 3 problems of this template for an
% example.
%
\newenvironment{homeworkProblem}[1][-1]{
    \ifnum#1>0
        \setcounter{homeworkProblemCounter}{#1}
    \fi
    \section{Problem \arabic{homeworkProblemCounter}}
    \setcounter{partCounter}{1}
    \enterProblemHeader{homeworkProblemCounter}
}{
    \exitProblemHeader{homeworkProblemCounter}
}

%
% Homework Details
%   - Title
%   - Due date
%   - Class
%   - Section/Time
%   - Instructor
%   - Author
%

\newcommand{\hmwkTitle}{Project 1}
\newcommand{\hmwkDueDate}{April 7, 2017}
\newcommand{\hmwkClass}{Calculus II}
% \newcommand{\hmwkClassTime}{Section A}
\newcommand{\hmwkClassInstructor}{Adrian Ulises Soto}
\newcommand{\hmwkAuthorName}{\textbf{Alejandro Olivares Arteaga -- A01337525}
  \and \textbf{Sergio Ugalde Marcano -- A01336435} \and \textbf{Miquel Alcobé
    Gómez -- A01336664
} \and \textbf{Jorge Constanzo De la Vega Carrasco -- A01650285
}  }

%
% Title Page
%

\title{
    \vspace{2in}
    \textmd{\textbf{\hmwkClass:\ \hmwkTitle}}\\
    \normalsize\vspace{0.1in}\small{Due\ on\ \hmwkDueDate\ }\\
    \vspace{0.1in}\large{\textit{\hmwkClassInstructor\ \hmwkClassTime}}
    \vspace{3in}
}

\author{\hmwkAuthorName}
\date{}

\renewcommand{\part}[1]{\textbf{\large Part \Alph{partCounter}}\stepcounter{partCounter}\\}

%
% Various Helper Commands
%

% Useful for algorithms
\newcommand{\alg}[1]{\textsc{\bfseries \footnotesize #1}}

% For derivatives
\newcommand{\deriv}[1]{\frac{\mathrm{d}}{\mathrm{d}x} (#1)}

% For partial derivatives
\newcommand{\pderiv}[2]{\frac{\partial}{\partial #1} (#2)}

% Integral dx
\newcommand{\dx}{\mathrm{d}x}

% Alias for the Solution section header
\newcommand{\solution}{\textbf{\large Solution}}

% Probability commands: Expectation, Variance, Covariance, Bias
\newcommand{\E}{\mathrm{E}}
\newcommand{\Var}{\mathrm{Var}}
\newcommand{\Cov}{\mathrm{Cov}}
\newcommand{\Bias}{\mathrm{Bias}}


\graphicspath{ {Img/} }
\usepackage{float}
\begin{document}

\maketitle

\pagebreak

\begin{homeworkProblem}
Choose a profile picture, which can be scanned from a drawing or desinged entirely on a computer. Each drawing must be
unique and must be approved before proceeding to continue the project. The drawing must have an axis of symmetry.
The design must be such that an application of Simpson's rule can be used to approximate the several integrals. Pixelated
desings are also possible. The boundary must be a simple closed curve with (no
holes, no self-crossing).\\\\
\textbf{Solution}

We chose an figure that when revolved, results in an innovative approach to
lifesavers, as it provides a  conforable space for placing one's arms and safely
placing items that may be useful to one's survival.
Our basic design is the following:
\begin{figure}[H]
  \centering
  \includegraphics[width=\textwidth, height=12cm, keepaspectratio]{Fixed1_2}
  \caption{Cross section design}
  \label{fig1}
\end{figure}
\end{homeworkProblem}

\begin{homeworkProblem}
  Choose a particular scale, origin and coordinate axes for your drawing.
 \\\\
  \textbf{Solution}
  Our coordinate axis is the following:
\begin{figure}[H]
  \centering
  \includegraphics[width=\textwidth, height=12cm, keepaspectratio]{Fixed1_2}
  \caption{Cross section with axis}
  \label{fig2}
\end{figure}
\end{homeworkProblem}

\begin{homeworkProblem}
  Numerically, and an approximation to the area of your drawing using Simpson's rule and the Trapezoidal rule. If your
drawing allows it, and an exact answer also.
 \\\\
\textbf{Solution}
First we present the samples and their respective heights:
\begin{figure}[H]
  \centering
  \includegraphics[width=\textwidth, height=12cm, keepaspectratio]{Fixed2_1}
  \caption{Cross section with measurements}
  \label{fig3}
\end{figure}

\textbf{Trapezoidal rule}\\
  By definition, we know that the Trapezoid rule has the following form:
  \begin{equation}
    \int_{a}^{b}f(x)dx\approx\frac{h}{2}[f(x_0)+2f(x_1)+f(x_2)+fx(_3)+\dots+2f(x_{n-1})+f(x_n)]
  \end{equation}
  Thus, substituting our measurements we get:
\begin{align*}
  \int_{a}^{b}f(x)dx&\approx\frac{3}{2}[21+2\times16.5+19+2\times19.5+19.5+\\
                    &2\times19.5+19.5+2\times19.5+19+2\times16.5+21]\\
    &\approx\frac{3}{2}[302]\\
    &\approx 453
   \end{align*}

   This the approximation of the area via the trapezoid rule is $Area \approx 453cm^2$

  \textbf{Simpon's method}
  \\
  By definition, we know that Simpon's method has the following form:
  \begin{equation}
    \int_{a}^{b}f(x)dx\approx\frac{3}{h}[f(x_0)+4f(x_1)+2f(x_2)+4fx(_3)+\dots+4f(x_{n-3})+2f(x_{n-2})+4f(x_{n-1})+f(x_n)]
  \end{equation}
  And substituting with our measurements:

  \begin{align*}
    \int_{a}^{b}f(x)dx&\approx\frac{3}{3}[21+4\times16.5+2\times19+4\times19.5+2\times19.5\\
    &\approx+4\times19.5+2\times19.5+4\times19.5+2\times19+4\times16.5+21]\\
    &\approx\frac{3}{3}[562]\\
    &\approx 562
   \end{align*}

   This the approximation of the area via Simpon's method is $Area \approx 562cm^2$

\end{homeworkProblem}

\begin{homeworkProblem}
 Numerically, and an approximation of the perimeter of your drawing approximating it using straight lines and adding the
 length.
 \\\\
 \textbf{Solution}\\
 From Figure \ref{fig3}, we can see that from our 16 measurements, we can approximate
 the perimeter as the sum:
 \[
   P = 21 + 7 + 3.5 + 3 + 12 + 3 + 3.5 + 7 + 21 + 7 + 3.5 + 3 + 12 + 3 + 3.5
   + 7
 \]
 Which gives us a result of: $P \approx 120cm$
\end{homeworkProblem}

\begin{homeworkProblem}
  Choose an axis parallel to the axis of symmetry of your figure on which you will revolve your figure to produce a life saver
  with your figure as profile.\\\\
  \textbf{Solution}\\
  We have choosen the axis as the axis of revolution $x = 0$, therefore according to our diagram we have an
  inner radius of 30cm and an outer radius  of 60cm.
\end{homeworkProblem}

\begin{homeworkProblem}
  Investigate Pappus' Theorem for solids of Revolution, and use it to
  \begin{enumerate}[label=\alph*)]
  \item Compute the volume of the resulting solid of revolution.
  \item Compute the surface area of the resulting solid of revolution
  \end{enumerate}
\\
  \textbf{Solution}\\
  \textbf{Volume of solid of revolution}\\
  From Pappus, we know that:
  \begin{align*}
    V &= 2 \pi r A
  \end{align*}
  Where $V$ is the volume, $r$ is the distance from the axis to the centroid and $A$ is the area of
  the region.
  Therefore, substituting, we get
  \begin{align*}
    V &\approx 2 \pi\times 45\times562\\
      &\approx 158901.756
  \end{align*}
  Thus the volume is approximately equal to $158901.8cm^3$\\
  \textbf{Surface area}\\
  From Pappus, we know that
  \begin{align*}
    S &= 2 \pi r L
  \end{align*}
  Where $S$ is the surface area, $r$ is the distance from the axis to the centroid and $L$ is the perimeter of
  the region.
  Therefore, substituting, we get
  \begin{align*}
    S &\approx 2 \pi 45 \times 120\\
      &\approx 33929.2
  \end{align*}
  Thus the surface is approximately equal to $33929.2cm^2$\\
\end{homeworkProblem}

\begin{homeworkProblem}
 Investigate Arquimedes' Theorem to And the buoyance force of the life saver assuming it is completely submerged in water,
 and is filled with air.\\\\

 \textbf{Solution}
 According to Archimedes' buoyancy principle, the  buoyant force is defined as:
 \begin{align*}
  \text{Buoyant Force} &= \text{weight of displaced liquid}\\
   F &= mg = \rho V g
 \end{align*}
 Where $m$ is the mass of the displaced liquid, $g$ is the force of gravity, V
 is the volume of the displaced liquid and \rho is the density of the liquid.

 Therefore, substituting (assuming a density of seawater of 1.025 kg/L and a
 value for $g$ of 9.81) we have
 \begin{align*}
  F &\approx  1.05 \times 562 \times 9.81\\
    &5788.881
 \end{align*}

 Thus the buoyant force is equal to 5788.881 Newtons.
\end{homeworkProblem}

\begin{homeworkProblem}
  Using Simpson's rule, and the volume of the solid of revolution without using
  Pappus' Theorem.\\\\
  \textbf{Solution}
  In order to find the volume of revolution using Simpon's rule, we are going to
  use the shell method.

  First we find the volume of a single cylinder (at $x_3$):
  \begin{align*}
   V &= 2 \pi (x) (f(u) - f(b)) \Delta x
  \end{align*}
  Where $x$ is the radius, $u$ is the upper
  bound, $b$ is the lower bound and \Delta x the thickness of the shell.

  Substituting we have
  \begin{align*}
    V &\approx 2 \ pi (39) (19.5) (3)\\
      &\approx 14335.08
  \end{align*}

  And an approximation to the volume would be:
  \begin{align*}
    V &\approx \sum_{j=0}^{10}2 \pi (x_j) (f(u_j) - f(b_j)) \Delta x
  \end{align*}

  Which would give us a result of:
  \begin{align*}
    V \approx & 2 \pi \times 3 \times (33\times16.5+36\times19+39\times19.5+42\times19.5+45\times19.5+48\times19.5+\\
    &51\times19.5+54\times19+57\times16.5+60\times21)\\
      \approx &166677.198
  \end{align*}

  Thus our approximations yields a volume of $166677.198cm^3$

\end{homeworkProblem}


\begin{homeworkProblem}
  Make a real design of your life saver. You may 3d print it if you wish.
\begin{figure}[H]
  \centering
  \includegraphics[width=\textwidth, height=12cm, keepaspectratio]{3d}
  \caption{3D Design}
  \label{fig3}
\end{figure}


\end{homeworkProblem}

\end{document}
